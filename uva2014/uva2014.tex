\documentclass[11pt]{beamer}

\usepackage{url}
\usepackage{tikz}
\author{Armijn Hemel, MSc\\Tjaldur Software Governance Solutions}
\title{Binary Analysis}
\date{October 29, 2014}

\begin{document}

\setlength{\parskip}{4pt}

\frame{\titlepage}

\frame{
\frametitle{About Armijn}

\begin{itemize}
\item using Open Source software since 1994
\item MSc Computer Science from Utrecht University (The Netherlands)
\item created original prototype for NixOS
\item core team \url{gpl-violations.org} from 2005 - May 2012
\item owner Tjaldur Software Governance Solutions
\item European coordinator Linux Defenders at Open Invention Network
\end{itemize}
}

\frame{
\frametitle{Today's topic}
Today I will talk about (automatically) analysing binary files, plus tell you why I do this. I will cover:

\begin{itemize}
\item consumer electronics supply chain issues
\item some current technologies for binary analysis I (co-)developed
\item new exciting projects that \textit{you} can work on
\end{itemize}

Let's keep this interactive as much as possible. Ask questions. Make remarks.
}

\frame{
\frametitle{Why analyse binaries}
Contrary to what many people in the software industry want to believe most software is distributed in binary form:

\begin{itemize}
\item EXE, RPM, DEB
\item embedded devices
\item apps
\item \dots
\end{itemize}

Source code really is the exception!
}

\frame{
\frametitle{Supply chains}
Marketing and rebranding is very common in many industries:

\begin{itemize}
\item food industry
\item soap
\item consumer electronics
\end{itemize}

Know your supply chains and you can save a lot of money!
}

\frame{
\frametitle{Supply chains in the electronics industry}
Devices are not made in the way that many people think. Instead, a (long) supply chain of companies is responsible for a single product:

\begin{itemize}
\item chipset manufacturer
\item original design manufacturer (ODM)
\item SDK vendor
\item 3rd party vendors (hardware drivers, etc.)
\end{itemize}

The name on the box doesn't mean anything.
}

\frame{
\frametitle{Consumer electronics: the truth}
Almost everything is purchased pre-fab in Asia. Making everything yourself is commercial suicide:

\begin{itemize}
\item extremely thin margins
\item cut throat competition
\item quality is less important than price
\item ``cowboys''
\end{itemize}

It's like Nike: don't do any production, just marketing and sales.
}

\frame{
\frametitle{}
  \begin{tikzpicture}[remember picture,overlay]
    %\node[yshift=0.3cm] at (current page.center)
    \node[yshift=-0.3cm] at (current page.center)
    {
      \pgfimage[width=\linewidth]{alibaba}
    };
  \end{tikzpicture}
}

\frame{
\frametitle{Problems observed in consumer electroncis}
Razor thin margins and high pressure cause people to take shortcuts:

\begin{itemize}
\item copyrights
\item security
\item hardware components (out of scope)
\end{itemize}

Because more software is commoditized because of open source chipset manufacturers are providing more and more standard solutions, cutting margins of ODMs. Market pressure from customers drives prices down more.
}

\frame{
\frametitle{Experiences with Asia regarding copyright}

Barriers:

\begin{itemize}
\item copyright works differently
\item cultural barriers
\item NDAs were mandatory for years, now source code is demanded
\item widely used open source licenses like GPLv2 are all in English
\item open source code (such as GPL) is seen as ``public domain''
\end{itemize}
}

\frame{
\frametitle{Experiences with Asia regarding copyright (2)}

\begin{itemize}
\item low risk getting caught, so why bother?
\item ``best practices'' are not applied: development processes are often quite sloppy
\item everything is short term
\item engineers switch companies fast, especially in Taiwan
\item competitors also don't follow licenses and ``save'' money
\end{itemize}
}

\frame{
\frametitle{Experiences with security}
Security is not a feature, so there are no tests done for it.

It is simply not on their radar \textit{at all}.
}

\frame{
\frametitle{Problem: how to discover issues}
We have all these devices. How are we going to find out what is inside a device and whether or not it is open source software, or is insecure?

What are \textit{your} thoughts?
}

\frame{
\frametitle{Firmware analysis}
Firmwares seem like opaque blobs, but actually there is structure:

\begin{itemize}
\item file systems
\item compressed files
\item archives
\end{itemize}

First look for headers/footers, carve out file systems, unpack them, then recursively scan files that were unpacked.

Observed issues: vendor specific modifications to file systems, obfuscation/encryption
}

\frame{
\frametitle{Analysing individual binaries}
Given a single binary how would you try to find out:

\begin{itemize}
\item contents (what open source software was used)
\item possible versions of open source software
\end{itemize}

\textit{Your} thoughts!
}

\frame{
\frametitle{Drawbacks of decompilation}

\begin{itemize}
\item various architectures
\item different compiler settings
\item possibly illegal in some jurisdictions
\item binaries are often stripped
\end{itemize}
}

\frame{
\frametitle{Fingerprinting!}
It is easier to fingerprint binaries using:

\begin{itemize}
\item string constants
\item function/method names
\item variable names
\end{itemize}
}

\begin{frame}[fragile]
\frametitle{String constants}
\begin{verbatim}
...
 } else {
    printf("%s %s: status is %x, should never happen\n",
           inst->prog, inst->device, status);
    status = EXIT_ERROR;
 }
 ...
\end{verbatim}

These are not stripped by the compiler and often occur in the exact same order as in the source code. This makes it ideal for fingerprinting.
\end{frame}

\frame{
\frametitle{Function names/variable names}
In dynamically linked ELF binaries it is possible to easily get lists of locally defined functions and variable names. Great for fingerprinting!

In Java and Dalvik binaries it is trivial to get this information.
}

\frame{
\frametitle{Match with information from source code}
A lot of open source code is reused unmodified, or with very few modifications. The LTSI project (Linux Foundation) showed that around 98\% of the Linux kernel code in Android handsets is not modified and changes are minimal (mostly driver code).

Many other components are not changed at all. Consumer electronics mostly uses open source, and you can download \& process that code easily.
}

\frame{
\frametitle{Extracting string constants and other identifiers}
There are great open source tools that can help extracting identifiers

\begin{itemize}
\item \texttt{xgettext} for string constants
\item \texttt{ctags} for function names and variable names
\end{itemize}

Results might not be 100\% perfect, but this works good enough.

You can store these results with other metainformation (file names, package names, version names, licenses)
}

\frame{
\frametitle{Match and report}
Now it is almost trivial to find out what package was used in a binary

\begin{enumerate}
\item extract identifiers from a binary
\item look up identifiers in a database containing identifiers extracted from source code
\item statistics, matching, reporting
\end{enumerate}

This works \textit{extremely} well: using just string constants I can often tell you which version was used.
}

\frame{
\frametitle{Binary Analysis Tool}
Binary Analysis Tool (or: BAT) is a lightweight tool under an open source license that automates binary analysis.

\begin{itemize}
\item demystify binary analysis by codifying knowledge
\item make it easier to have reproducable results
\item common language for binary analysis
\end{itemize}

BAT is a generic framework for binary analysis. Until now focus of BAT was primarily on software license compliance.

BAT has been in development since late 2009 and getting reasonably good.

Funding came from Linux Foundation, NLnet Foundation, Cisco, as well as my customers.
}

\frame{
\frametitle{Binary Analysis Tool result demo}
Let's look at results of one scan of a (crappy) consumer electronics router.
}

\frame{
\frametitle{Combining fingerprinting with metadata}
Right now I mostly deal with license compliance issues, but there are other applications as well.

Discovering security issues is by far the most important one (next talk, after the break).
}

\frame{
\frametitle{BAT work that needs to be done}
BAT is not perfect and there is work that needs to be done:

\begin{itemize}
\item more file system support (UBIFS)
\item better GUI
\item support for QNX ELF files
\item detection of compiler optimization options
\item detecting if ``broken \texttt{sstrip}'' was used
\end{itemize}

and there are many more research ideas I have. If you are still looking for RP1 or RP2 tasks, talk to me!
}

\frame{
\frametitle{Questions?}
}

\frame{
\frametitle{Contact}

\begin{itemize}
\item \url{armijn@tjaldur.nl}
\item \url{http://www.tjaldur.nl/}
\item Binary Analysis Tool: \url{http://www.binaryanalysis.org/}
\end{itemize}
}

\end{document}
