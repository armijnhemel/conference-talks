\documentclass[11pt]{beamer}

\usepackage{url}
\usepackage{tikz}
\author{Armijn Hemel\\Tjaldur Software Governance Solutions}
\title{}
\date{February 13, 2012}

\begin{document}

\setlength{\parskip}{4pt}

\frame{\titlepage}

\frame{
\frametitle{About Armijn}

\begin{itemize}
\item using Open Source software since 1994
\item MSc Computer Science from Utrecht University (The Netherlands)
\item core team \url{gpl-violations.org} since 2005
\item ex-board member at NLUUG (\url{http://www.nluug.nl/})
%\item sysadmin, developer and consultant at Loohuis Consulting (2006 - May 2011)
\item owner Tjaldur Software Governance Solutions
\end{itemize}
}

\frame{
\frametitle{Subjects}

\begin{itemize}
\item very brief overview of license violations
\item problems in binary code clone detection
\item open questions in binary code clone detection
\end{itemize}
}

\frame{
\frametitle{License enforcement}

\begin{itemize}
\item Europe (Germany, France) \& USA
\item focus is on GPLv2 and LGPLv2/2.1
\item done by companies (Nokia, Red Hat) and individual developers and projects (Harald Welte, BusyBox, XviD, etc.)
\end{itemize}

It is about copyright, not about patents!
}

\frame{
\frametitle{gpl-violations.org}
Founded in 2004 by Harald Welte (copyright holder in the Linux kernel) to take on GPL license violations by:

\begin{itemize}
\item education
\item documentation
\item legal action
\end{itemize}

I have been active with \url{gpl-violations.org} since 2005.

So far we've had several hundred cases (most of them settled) and fixed many more using informal pressure.
}

\frame{
\frametitle{How gpl-violations.org works}

\begin{enumerate}
\item we get a report via private email, public mailing list, chat, rumours, SMS, or our own research
\item if there is reasonable doubt about compliance of a device we do a test purchase to confirm the violation
\item if we confirm a violation we send a ``cease and desist''
\end{enumerate}

There are many false reports: a lot of people don't understand the license(s).

Our main focus is on consumer electronics (one of the biggest markets out there).
}

\frame{
\frametitle{Consumer electronics: the truth}
Almost everything is purchased. Making everything yourself is commercial suicide:

\begin{itemize}
\item extremely thin margins
\item cut throat competition
\item quality is less important than price
\item (ultra) short term thinking: companies don't know if they will still be in business in 6 months from now
\item ``cowboys''
\end{itemize}

It's like Nike: don't do any production, just marketing and sales.

In my experience typically more than 95\% (or more) is reuse of open source software (with/without modifications)
}

\frame{
\frametitle{}
  \begin{tikzpicture}[remember picture,overlay]
    %\node[yshift=0.3cm] at (current page.center)
    \node[yshift=-0.3cm] at (current page.center)
    {
      \pgfimage[width=\linewidth]{alibaba}
    };
  \end{tikzpicture}
}

\frame{
\frametitle{Problem source: supply chain}
License violations are often a direct result of a mistake made in the supply chain:

\begin{itemize}
\item chipset vendors
\item board makers
\item SDK (``Software Development Kit'') vendor
\item reference design makers
\item product customizers
\item ``labellers''
\end{itemize}

The ``labellers'' get sued and are responsible, even though they add/modify the least amount of code!
}

\frame{
\frametitle{Industry responses to enforcement}

\begin{itemize}
\item extreme levels of frustration (problem doesn't go away by throwing money at it)
\item they don't care about licenses, they just want to sell a product. Licenses are a nuisance that needs to be dealt with.
\item a single enforcement case will make no change to the market (it is too big: a single company getting in trouble is not significant to push for change)
\item no ill will. Companies want to fix it and there is a need for tools (cheap, or free) to do ``due dilligence''
\end{itemize}
}

\frame{
\frametitle{Tools}
Apart from the obvious ``industry standard ''tools that solve some problems Tjaldur Software Governance Solutions has worked on tools to help solving specific problems in this field.

Goal: let companies do checks themselves, increasing quality and lowering costs.

\begin{itemize}
\item Binary Analysis Tool (Apache 2 license, freemium model)
\item license scanning tools (leveraging existing tools like Ninka and FOSSology)
\item long term: build system integration (preliminary work has been done)
\end{itemize}
}

\frame{
\frametitle{Binary Analysis Tool}

\begin{itemize}
\item generic extensible pluggable framework for analysing binaries
\item binary code clone detection using string comparisons: first extract string constants from the binary, compare it with a large database of data extracted from source code, finally assign a score to packages based on matches
\end{itemize}

Demo later this week.
}

\frame{
\frametitle{Academic research}

There is a lot of unclarity about licenses and software provenance, and academia can help here. I'm trying to do my share, team up with researchers for research:

\begin{itemize}
\item ``Finding Software License Violations Through Binary Code Clone Detection'' (Mining Software Repositories 2011) - some results have been integrated into Binary Analysis Tool
\item ``What Goes into an Executable? Identifying a Binary's Sources by Tracing Build Processes'' (sent to WCRE 2011 and ICSE 2012, unfortunatey rejected) - I'm preparing tooling that implements this
\item future research (your name here)
\end{itemize}
}

\frame{
\frametitle{Open questions/problems}

\begin{itemize}
\item detecting obfuscated code in binaries (when basic string comparisons simply aren't enough)
\item detecting language embedding (interpreters, DSLs) in binaries (if they have been compiled)
\item correlating binary code and source code (solved for source to binary using tracing, not from binary to source side)
\item complete provenance of binary and source code files, down to the level of single commits (example: individual Git commits) because snapshots from DVCS (like Git) are rapidly replacing normal releases
\item reducing false positives in detection: false claims can lead to counter lawsuits, with significant risks
\end{itemize}
}

\end{document}
