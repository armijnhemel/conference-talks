\documentclass[11pt]{beamer}

\usepackage{url}
\usepackage{tikz}
\author{Armijn Hemel, MSc\\Tjaldur Software Governance Solutions}
\title{Introducing the Binary Analysis Tool}
\date{November 14, 2013}

\begin{document}

\setlength{\parskip}{4pt}

\frame{\titlepage}

\frame{
\frametitle{About Armijn}

\begin{itemize}
\item using Open Source software since 1994
\item MSc Computer Science from Utrecht University (The Netherlands)
\item core team \url{gpl-violations.org} from 2005 - May 2012
\item owner Tjaldur Software Governance Solutions
\end{itemize}
}

\frame{
\frametitle{Binary Analysis Tool}
Binary Analysis Tool (or: BAT) is a lightweight tool under an open source license that automates binary analysis.

\begin{itemize}
\item demystify compliance engineering by codifying knowledge
\item make it easier to have reproducable results
\item common language for binary analysis
\item only analyses binaries, but draws no legal conclusions
\end{itemize}

Although BAT is a generic framework for binary analysis my focus is on software license compliance.

Important: a license violation is not a technical issue, but a legal issue. Technical measures are only used to obtain evidence.

BAT 15 was released on October 10 2013.
}

\frame{
\frametitle{Why analyse binaries?}

\begin{itemize}
\item software is often supplied in binary form by vendors (on a device/CD/DVD/flash chip/download/app). Sometimes source code is supplied and if you're lucky it might match the binary.
\item even between companies (for example in a supply chain) software is often shipped as binaries
\end{itemize}

Shipping software as source code is the \textit{exception}.

If you pass on binary software that you get from upstream (for example 3rd party software in a product) you \textit{have} to know what you ship! This means you have to analyse the binaries.
}

\begin{frame}[fragile]
\frametitle{What's in this blob?}

{\scriptsize\color{blue}{
\begin{verbatim}
00000000  50 4b 03 04 14 00 00 00  08 00 29 52 57 3c fa c0  |PK........)RW<..|
00000010  03 a7 26 9e 16 01 f4 ae  19 01 15 00 00 00 76 31  |..&...........v1|
00000020  2e 31 2e 31 2e 31 37 5f  53 4d 43 5f 61 6c 6c 2e  |.1.1.17_SMC_all.|
00000030  65 78 65 ec 3a 6d 78 53  55 9a f7 26 69 9a 42 ca  |exe.:mxSU..&i.B.|
00000040  0d d0 38 65 69 30 60 50  94 96 56 43 91 98 06 03  |..8ei0`P..VC....|
00000050  92 18 9f e1 e3 d6 c8 4d  03 f4 03 69 6b b8 a3 88  |.......M...ik...|
00000060  78 2f 83 da 76 c3 a6 d9  6d 7a 37 0f 38 8b 33 ae  |x/..v...mz7.8.3.|
00000070  33 ce d0 89 ee 8a f8 38  ae 3a 88 1f 30 61 c2 52  |3......8.:..0a.R|
00000080  3a ea 33 ac e3 02 0e 3c  b3 38 ea ee e9 a4 ce d4  |:.3....<.8......|
00000090  85 2d 01 0b 77 df f7 dc  f4 03 1c 67 66 9f fd db  |.-..w......gf...|
000000a0  ab 37 f7 9c f7 bc e7 fd  38 e7 bc 5f a7 ac 5c bb  |.7......8.._..\.|
000000b0  8b d1 33 0c 63 80 57 55  19 e6 00 a3 3d 5e e6 cf  |..3.c.WU....=^..|
000000c0  3f 67 e1 9d 72 fd 5b 53  98 d7 8b de 9f 7d 80 5d  |?g..r.[S.....}.]|
000000d0  f1 fe ec fb 22 9b 1e b5  6f d9 fa f0 03 5b 37 3c  |...."...o....[7<|
000000e0  64 df b8 61 f3 e6 87 25  fb fd 2d f6 ad f2 66 fb  |d..a...%..-...f.|
000000f0  a6 cd f6 e5 ab 83 f6 87  1e 6e 6e 59 50 5c 3c c9  |.........nnYP\<.|
00000100  91 a7 d1 fc c1 99 4b f6  d7 5e dd 3b f2 5e da f5  |......K..^.;.^..|
00000110  f2 de 6a f8 ae 7e e9 cd  bd f3 e0 9b fa c9 3b 7b  |..j..~........;{|
00000120  17 d2 fe 81 bd 9b e0 fb  eb 5d fb f6 56 52 dc d7  |.........]..VR..|
00000130  f6 7e 1f be 37 ee 7a 73  ef 2d f0 fd af 9f be be  |.~..7.zs.-......|
00000140  77 36 7c ef dd b4 31 82  74 46 64 e4 7d 0c b3 82  |w6|...1.tFd.}...|
00000150  35 30 43 1b fd 9e 31 b9  39 76 32 6b 64 98 2a 96  |50C...1.9v2kd.*.|
00000160  61 9a f4 14 76 a1 1b 7e  2c a8 38 ab 69 6f d1 fa  |a...v..~,.8.io..|
00000170  86 fc 9c 91 2f b3 c7 a0  8d c1 a3 a3 bf 96 7c df  |..../.........|.|
00000180  32 0a b7 8c 5b a3 c8 3d  2c b3 07 1b c7 59 e6 85  |2...[..=,....Y..|
00000190  5f e8 fe 82 55 fd 0b 1f  90 d3 a0 ff fa e1 05 52  |_...U..........R|
000001a0  cb 76 09 be 8b 1c ac 26  50 15 7b b5 60 f0 d8 41  |.v.....&P.{.`..A|
000001b0  fb 05 5b 9b 37 48 1b 18  e6 d5 19 1a 4d 04 6a 6b  |..[.7H......M.jk|
000001c0  30 8e 15 fc bf 40 43 63  9a 37 c3 4f 13 ab 1d 90  |0....@Cc.7.O....|
000001d0  a6 af e0 a5 17 6c 7d 74  eb 46 68 53 5d 41 67 e6  |.....l}t.FhS]Ag.|
000001e0  38 7c f7 7c 95 de ff 4d  d9 89 67 e2 99 78 26 9e  |8|.|...M..g..x&.|
000001f0  89 67 e2 99 78 26 9e 89  67 e2 99 78 26 9e 89 e7  |.g..x&..g..x&...|
00000200  ff fb ac 51 06 62 03 e6  a0 f3 74 30 b6 72 58 8d  |...Q.b....t0.rX.|
00000210  44 01 24 ea 83 22 1b 81  46 54 95 4d aa b5 64 e9  |D.$.."..FT.M..d.|
00000220  52 46 59 69 8e e5 d4 84  ef 7c 2f 3b a4 aa 2a d7  |RFYi.....|/;..*.|
00000230  b9 03 86 83 c1 a0 a8 0b  f2 aa b5 14 30 dc 99 84  |............0...|
00000240  2f 27 35 87 10 68 00 58  19 ce ca b9 bf 94 ee de  |/'5..h.X........|
00000250  71 e7 ca e0 5d 7e d9 29  28 df b6 e8 2f b4 ba 1a  |q...]~.)(.../...|
00000260  cb 12 f5 c3 89 7a d3 36  bb 62 8c 1d 9d ca fa 86  |.....z.6.b......|
\end{verbatim}
}}

\end{frame}

\frame{
\frametitle{Binary analysis}
A binary usually looks like a blob with random data. Often there is a structure,
with embedded file systems or compressed files that can ``easily'' be recognised.
}

\frame{
\frametitle{Analysis steps}
Steps to determine if a binary contains a particular source code:

\begin{enumerate}
\item extract binary files from blobs (firmwares, installers, etc.) recursively (if needed)
\item extract identifiers (strings, function names, variable names, etcetera) from binary files and compare these to (publicly available) source code
\item use other information like file names, presence of other files, package databases, etcetera, for circumstantial evidence
\end{enumerate}
}

\frame{
\frametitle{``Ducktyping''}
``If it looks like a duck and quacks like a duck, it is probably a duck''

If you can relate many strings, function names, variable names, and so on from a binary file to source code it becomes statistically hard to deny (re)use of a certain software.

Often it is possible to match hundreds or even thousands of strings, function/method names or variable names.
}

\frame{
\frametitle{Drawbacks of manual inspection}
Checking can be done by hand using standard Linux tools, but there are drawbacks:

\begin{itemize}
\item limited by the knowledge of the engineer
\item time consuming (so expensive)
\item easy to overlook things
\end{itemize}

So you really want to automate this! BAT can do this for you.
}

\frame{
\frametitle{Place of BAT in an open source compliance process}
BAT is not meant as a replacement of a source code scanner, because it focuses on different problems.

BAT is useful when:

\begin{itemize}
\item you get binaries, but no source code and want to know what could be in there
\item you get binaries and source code, but don't know if binaries and sources match
\item you want to know how binaries interact (for example linking), which a source code scanner cannot tell you
\end{itemize}
}

\frame{
\frametitle{Inner workings of BAT}

\begin{enumerate}
\item discovery of offsets of known file systems and compressed files, verification of file types and unpacking of found file systems and compressed files (recursively)
\item check each unpacked file, like identifier search or ELF linking verification
\item reporting, generating pictures, etcetera
\end{enumerate}
}

\frame{
\frametitle{BAT modules}

BAT is extremely modular and it comes with several modules:

\begin{itemize}
\item unpacking over 30 file systems and compressed files
\item report on common properties (file type, size, etcetera)
\item search for license markers and identifiers
\item advanced string identifier search
\item dynamic ELF linking verification
\item kernel module analysis
\item many more
\end{itemize}
}

\frame{
\frametitle{Advanced identifier search/ranking}
Most advanced check in BAT extracts string constants, function/method names, variable names, etcetera, from binaries and compares them with a large database of strings, function/method names, etcetera extracted from source code:

Currently over 170,000 packages from GNU, GNOME, KDE, Samba, Debian, Savannah, FedoraHosted, Linux kernel, Maven, F-Droid, \dots

Database is not part of BAT, but only available as a subscription, or you can ``roll your own''.

Algorithm has been published at the Mining Software Repositories 2011 conference and most scripts to create the database are open.
}

\frame{
\frametitle{Inner workings of BAT ranking}
BAT ranking algorithm uses a database where data is extracted from \textit{source code}:

\begin{itemize}
\item string constants (using \texttt{xgettext} and regular expressions for some Linux kernel code)
\item function names (C) and method names (Java) (using \texttt{ctags})
\item variable names and Linux kernel symbols (C), field names and class names (Java) (using \texttt{ctags})
\item licenses (if enabled) (using Ninka and FOSSology)
\item Linux kernel module info (using regular expressions)
\item various characteristics of the file (SHA256 checksum, etc.)
\end{itemize}

Core of algorithm uses string constants (bulk of the usable information in the binary), rest of information is used to verify/strengthen the findings.
}

\begin{frame}[fragile]
\frametitle{String constant example}

\begin{verbatim}
...
 } else {
    printf("%s %s: status is %x, should never happen\n",
           inst->prog, inst->device, status);
    status = EXIT_ERROR;
 }
 ...
\end{verbatim}
\end{frame}

\frame{
\frametitle{String extraction from binaries}
From each binary that has not yet been discarded (graphics, video, audio, resources files and text files are not interesting so ignored) string constants are extracted using \texttt{strings}.

String constants are used for fingerprinting because they are not discarded by the compiler.

Some preprocessing steps can be used to increase quality of the strings extracted (to avoid false positives and get better scan results).
}

\frame{
\frametitle{Scoring (1)}
Each binary file is sorted into a family of languages:

\begin{itemize}
\item C (C/C++/QML/etc. + unknown binaries)
\item Java (JDK/Dalvik/Scala/etc.)
\item C\#
\item ActionScript
\end{itemize}

Reason is that strings that are very insignficant in one family could be very significant in another and vice versa.

Drawback: language embedding (specifically .NET) is at odds with this. For most systems (Java, embedded Linux) this is actually not much of an issue.
}

\frame{
\frametitle{Scoring (2)}
Each string constant is compared to the database. If a match is found in the database a score is assigned to that string.

The score for a unique string (single package) is the length of the string.

If it is not unique the score very rapidly drops depending on in how many different packages it can be found.

If there is \textit{cloning} the string is assigned to a package using an algorithm that picks the most promising package.
}

\frame{
\frametitle{Cloning}
Sometimes the wrong package will be detected, but this reflects how open source works!

\begin{itemize}
\item software reuse: code is ``cloned'' between packages. Some packages are forked and incorporated (partially) into others (in Java more so than in C code).
\item packages are renamed for various reasons. For example in Debian: \texttt{httpd} is renamed to \texttt{apache2}, Firefox is Iceweasel, and so on.
\end{itemize}

BAT tries to take alternatives and ``cloning'' into account.
}

\frame{
\frametitle{Some BAT statistics}
\begin{itemize}
\item quad core Intel(R) Core(TM) i7-3770 CPU (3.40GHz) with Hyper Threading, so 8 threads in total
\item 6 GiB DDR3 RAM
\item Samsung 840 PRO SSD
\item SATA3 disk
\item stock Fedora 19, with one package from RPM Fusion, \texttt{/tmp} on tmpfs disabled
\item BAT 15
\end{itemize}

Hardware costs (late 2012): about 930 Euro

All tests were done on a freshly booted system. Database resides on SSD. Firmwares are unpacked on SSD, extra tmpfs to speed up LZMA unpacking, \texttt{compress} unpacking and DEX scanning. Final result tarball written to SATA disk, unpacked data excluded.
}

\frame{
\frametitle{Example: Trendnet TEW-636ABP}
Filename: TEW-636APB-1002.bin

Size of the file is 4.0 MiB (complete flash dump). Total data after BAT unpacked it is 15 MiB.

In total 262 files were scanned:

\begin{itemize}
\item 89 ELF files
\item 48 GIF files
\item rest: empty files, HTML files, JavaScript, shell scripts and other files
\end{itemize}

Total scan time: 23s
}

\frame{
\frametitle{Example: ASUS O!Play Air}
Filename: HDP\_R3\_FW\_128\_PAL.zip

Size of the file is 24 MiB. Total data after BAT unpacked it is 272 MiB.

In total 1351 files were scanned:

\begin{itemize}
\item 96 ELF files
\item 90 PNG files
\item 346 XML files
\item 95 ASCII text files
\item rest: data files, scripts, other text files, others
\end{itemize}

Total scan time: 1m37s
}

\frame{
\frametitle{Example: ASUS PadFone 2}
Filename: JP\_PadFone2\_10\_4\_11\_13UpdateLauncher.zip

Size of the file is 780 MiB. Total data after BAT unpacked it is 3.5 GiB.

In total 56,000+ files were scanned:

\begin{itemize}
\item 36971 PNG files
\item 9 GIF files
\item 658 ELF files
\item 54 Android classes.dex files
\item 129 Android odex files
\item 191 Android resource files
\end{itemize}

Total scan time: 21m
}

\frame{
\frametitle{Demo: Trendnet TEW-636ABP}
Demo is walking through the results since a scan would take too long on this netbook.
}

\frame{
\frametitle{Database challenges}
There are challenges in creating a database:

\begin{itemize}
\item \textit{huge} amounts of data (and not getting less)
\item package names and file names are very important in BAT. DVCS like Git make software development more fluid.
\item cloning of software between packages.
%\item current implementation assigns non-unique matches to largest package, which is sometimes incorrect (for example: assign strings from \texttt{zlib} to \texttt{icedove}). This will be fixed in newer versions.
\end{itemize}

These challenges are not exclusive to BAT.
}

\frame{
\frametitle{Database quality}
Results of scanning are dependent on quality of the database: if only BusyBox is included everything will look like BusyBox. Making a good database is not easy:

\begin{itemize}
\item What to include?
\item What to exclude?
\item When is a package a new package?
\end{itemize}
}

\frame{
\frametitle{Other functionality in BAT}

\begin{itemize}
\item finding duplicate files in firmwares
\item finding leftover kernel modules (mismatching kernel version numbers)
\item ELF dynamic linking dependency inspection
\item (experimental) matching of results of binaries with source code archives
\item and more
\end{itemize}
}

\frame{
\frametitle{ELF dynamic linking}
Actual license of a binary is only determined at \textit{run time}. This is a
largely unresearched area.

BAT can give more information about how ELF binary files interact.
}

\frame{
\frametitle{ELF dynamic linking dependency graph}
  \begin{tikzpicture}[remember picture,overlay]
    %\node[yshift=0.3cm] at (current page.center)
    \node[yshift=-0.1cm] at (current page.center)
    {
      \pgfimage[width=\linewidth]{dot-busybox3}
    };
  \end{tikzpicture}
}

\frame{
\frametitle{Challenges}

\begin{itemize}
\item theory versus practice
\item more data
\item disk I/O
\item new file systems
\item compiler settings
\item LZMA unpacking
\item increasing size of firmwares
\item encryption/obfuscation
\item reporting
\end{itemize}
}

\frame{
\frametitle{Challenge: theory versus practice}
I often hear ``How hard can it be?''

\begin{itemize}
\item proper scanning takes a lot of bookkeeping
\item you have to deal with many exceptions and weird data (example: dangling JFFS2 inodes, corrupt archives, etc.)
\item non-standard/modified archives/file systems (SquashFS with LZMA)
\end{itemize}
}

\frame{
\frametitle{Challenge: more data}
There is more and more software written, published and reused. This leads to
larger databases to query.

Example: the Linux kernel has grown massively in the last 10 years.

Example: Android apps
}

\frame{
\frametitle{Challenge: disk I/O}
I/O can take quite a bit of total scanning time. Mostly this is:

\begin{itemize}
\item querying the database
\item LZMA unpacking (temporary files)
\end{itemize}

Many tricks are already used in BAT to decrease disk I/O. Smart system
setup seriously helps (using SSD, enough memory for tmpfs, etcetera).
}

\frame{
\frametitle{Challenge: file systems}
New file systems that are used, or variants of file systems are sometimes
difficult to unpack:

\begin{itemize}
\item YAFFS2 with different settings: lots of different versions float around
\item UBI \& UBIFS: \texttt{unubi} was removed from \texttt{mtdutils}
\item ext4 with extents and no ext2 compatibility
\item countless variations of SquashFS
\item vendor specific tweaks to normal file systems (for whatever reason)
\end{itemize}
}

\frame{
\frametitle{Challenge: compiler settings}
Some compiler settings move strings (currently extracted from
\texttt{.data}, \texttt{.rodata} and similar sections from the ELF file) to
other sections inside the ELF file, but I don't know (yet) when.

Also, sometimes function names are not found in the dynamic section.

Detecting these compiler settings from the generated byte code is future work.
}

\frame{
\frametitle{Challenge: LZMA unpacking}
LZMA does not have a single ``magic'' header. Lots of different valid headers
are possible (962,072,674,304 combinations possible for the first 5 bytes).

However, only a few are in widespread use.

Some filtering is done (based on information found ``in the wild'' and various
implementations), but the only way to be complete is by trying to unpack every
possible valid file. This can be very time consuming.

Being complete would cost too much time.
}

\frame{
\frametitle{Challenge: increasing size of firmwares}
Firmware updates of 1 GiB or more are no longer an exception.

Android file systems often contain two separate user lands (one with Google's
tools, one with tools that actually work).

After after unpacking there are easily over 50,000 files to scan.

Bigger sizes amplify challenges mentioned earlier.
}

\frame{
\frametitle{Challenge: encryption/obfuscation}
Some vendors encrypt or obfuscate firmwares, ranging from obvious (XOR) to
real encryption (AES).

BAT can handle a few XOR implementations right now (hard coded, experimental
feature, disabled by default). Cryptanalysis is currently not on the roadmap,
but could be interesting (technically and legally).
}

\frame{
\frametitle{Challenge: reporting}
BAT is pretty new technology and it is unclear what the most effective way to
interpret the information generated by BAT is.

This needs more eyeballs and review: for \textit{me} it is clear, but I want to
know opinions of others.
}

\frame{
\frametitle{Challenge: perfection}
``Perfect is the enemy of good''

Working with binaries means working with incomplete information and 100\%
accuracy is impossible to achieve...but I try.
}

\frame{
\frametitle{Upcoming functionality in BAT (next few months)}

\begin{itemize}
\item overcoming or avoiding challenges mentioned earlier
\item better GUI
\item deployment as a webservice
\item recreation of kernel configuration from binary kernel image and kernel modules (prototype works quite well)
\end{itemize}
}

\frame{
\frametitle{Looking further into the future}
There is a lot of information that can be extracted from binaries. Right now
only a tiny fraction of the information is used. Generated code for
example is not (yet) used.

There are other data sources that are still untapped like security information
bug reports, and so on. I will integrate more of these in the future.

I have ideas for at least another few years of coding.
}

\frame{
\frametitle{Questions?}
}

\frame{
\frametitle{Contact}

\begin{itemize}
\item \url{armijn@tjaldur.nl}
\item \url{http://www.tjaldur.nl/}
\item Binary Analysis Tool: \url{http://www.binaryanalysis.org/}
\end{itemize}
}

\end{document}
