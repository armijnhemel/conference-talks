\documentclass[10pt]{beamer}

\usepackage{url}
\author{Armijn Hemel}
\title{Android GPL license compliance}
\date{November 17, 2011}

\begin{document}

\setlength{\parskip}{4pt}

\frame{\titlepage}

\frame{
\frametitle{About Armijn}

\begin{itemize}
\item using Open Source software since 1994
\item MSc Computer Science from Utrecht University (The Netherlands)
\item core team \url{gpl-violations.org} since 2005
\item ex-board member at NLUUG (\url{http://www.nluug.nl/})
\item sysadmin, developer and consultant at Loohuis Consulting (2006 - May 2011)
\item May 2011 - present: owner Tjaldur Software Governance Solutions
\end{itemize}
}

\frame{
\frametitle{Talk overview}

\begin{itemize}
\item GPL enforcement
\item license violations in Android
\item license violation discovery
\item license violation resolution
\item license violation prevention
\item Q\&A
\end{itemize}

I will not talk about software patents. Focus of this talk is on GPL version 2.
}

\frame{
\frametitle{GPL enforcement}

\begin{itemize}
\item is real
\item is happening in Europe and US
\item happens mostly for GPLv2 and LGPLv2/2.1
\item is being done by both companies and individuals
\item is easy to avoid
\end{itemize}
}

\frame{
\frametitle{gpl-violations.org}
Harald Welte founded \url{gpl-violations.org} in 2004 to tackle GPL license violations through education, documentation and, as a last resort, legal action.

I joined in October 2005.

\url{gpl-violations.org} has solved several hundreds of cases through legal action and informal action.
}

\frame{
\frametitle{License violation life cycle}

Mistakes happen, but they need to be resolved.

\begin{enumerate}
\item understanding the problem
\item acknowledging the problem
\item fixing the problem
\item preventing the problem
\end{enumerate}
}

\frame{
\frametitle{Violations in Android}
There are two main areas where things go wrong:

\begin{itemize}
\item devices
\item Android marketplace
\end{itemize}

I will not talk (much) about applications in marketplaces, but focus on devices.

Note: it is a myth that Android user space does not contain GPL code. Google ships Android userspace without GPL code, but many manufacturers put GPL code back in (like BusyBox)!
}

\frame{
\frametitle{Notable examples}
Two cases of (alleged) GPL violations were very visible in the last year:

\begin{itemize}
\item HTC
\item Android tablet incompliance
\end{itemize}
}

\frame{
\frametitle{HTC}
HTC assembles software themselves, using a SDK from upstream (Qualcomm).

HTC has consistently stepped on people's toes:

\begin{itemize}
\item standard response: ``Please wait 60 to 90 days''
\item persistent, even after being asked to fix this, leading to bad PR
\end{itemize}

HTC needs to fix their software release processes.
}

\frame{
\frametitle{Android tablets}
Kernel developer Matthew Garrett got quite angry in late 2010 and assembled a list of Android tablets that are not in compliance.

The companies on his list don't do development, but only rebadge and resell.

Funny responses \url{gpl-violations.org} got:

\begin{itemize}
\item ``Our device is based on Android, not on Linux''
\item ``You can get the source code from Google''
\item ``We made some changes specifically for our tablet hardware, so you can't have the source code''
\item ``we think the kernel code is under GLPv3 so we have 60 days to fix the issue''
\end{itemize}

The problem is not these companies, but the whole supply chain that does not implement proper software governance.
}

\frame{
\frametitle{Discovering violations}

Important: violating a license is not a technical issue, but a legal issue. Technical measures (``GPL compliance engineering'') are only used to obtain evidence.

Two steps need to be taken:

\begin{enumerate}
\item documentation analysis
\item technical analysis
\end{enumerate}
}

\frame{
\frametitle{Documentation analysis}
The GPLv2 license needs:

\begin{enumerate}
\item copy of the license text
\item complete and corresponding source code for programs distributed under GPLv2, or
\item written offer for the complete and corresponding source code
\end{enumerate}

Points 1 and 3 can easily be checked by non-technical people.

Note: the ``legal information'' tab in the Android user interface \textit{usually} fullfills point 1.
}

\frame{
\frametitle{Technical analysis}
Technical analysis consists of two parts:

\begin{itemize}
\item determining the presence of GPL licensed software in binaries
\item check source code if it matches said binaries
\end{itemize}

Sometimes hardware needs to be modified to get access to the juicy bits (like adding a serial port).
}

\frame{
\frametitle{Analysing binaries}

\begin{itemize}
\item extract programs from binary blobs using various techniques
\item extract human readable strings from binary programs and match with source code
\item look at meta information (file names, package meta data, etc.)
\end{itemize}

If you can match a significant number of specific strings it becomes statistically hard to deny reuse of GPL licensed software.

You can do this by hand using many standard Linux tools, or use an automated tool like the Binary Analysis Tool.
}

\frame{
\frametitle{The Binary Analysis Tool}
The Binary Analysis Tool (BAT) is a tool that automates detection of Open Source software in a binary (firmwares, installers, etcetera).

Goals:

\begin{itemize}
\item demystifying compliance engineering by codifying processes
\item creating a reproduceable process and results
\item creating a common set of tools and language for looking at binaries
\item taking away excuses for companies (``it is difficult to find out'')
\end{itemize}

BAT is currently maintained by Tjaldur Software Governance Solutions and
can be found at \url{http://www.binaryanalysis.org/}.
}

\frame{
\frametitle{The Binary Analysis Tool - facts}
\begin{itemize}
\item Apache 2 licensed
\item lightweight
\item wraps around standard Unix tools plus a bit of custom code
\item spawned academic research (presented at Mining Software Repositories 2011 conference)
\item just examines binaries, but draws no legal conclusions from findings
\end{itemize}

Analysis of Android application files was added last month.

Want a demo? Ask me later in the hallway.
}

\frame{
\frametitle{Analysing source code}
\begin{enumerate}
\item find out what is in the binary
\item check for every binary that contains GPL licensed code if there is matching source code, under a GPL compatible license
\item check if there are no accidental leftover binaries without sources in the source code archive itself
\end{enumerate}
}

\frame{
\frametitle{Source code scanning versus binary scanning}

So what about using source code scanning tools to ensure the binary does not contain unwanted surprises?

Unless you build the binaries yourself using the provided source code you will \textit{never} be able to tell what is in a binary using only a source code scanning solution!

I am working on tools for combining license checking with information from build systems.
}

\frame{
\frametitle{Handling violation reports}
A problem is the way violation reports are handled, or rather: are not handled.

This leads to:

\begin{itemize}
\item angry people
\item bad PR
\item unnecessary misunderstandings
\item long term damage with upstream projects (HTC was told to ``wait 60 to 90 days'' when they needed help on LKML)
\end{itemize}
}

\frame{
\frametitle{Fixing communications}

\begin{itemize}
\item instruct support staff to hand off requests for source code. Don't let them just come up with something.
\item establish a separate point of contact regarding open source license questions, preferably a team (to avoid ``single point of failure'') and advertise this on your website/in documentation.
\item record all cases in a request tracker or CRM system, so there is a history.
\end{itemize}
}

\frame{
\frametitle{Fixing a violation}
Ideally:

\begin{itemize}
\item make complete and corresponding source code available
\item inform customers of their rights
\item take preventive measures so it never happens again
\end{itemize}

If this is not possible:

\begin{itemize}
\item stop distribution completely
\item take preventive measures so it never happens again
\end{itemize}
}

\frame{
\frametitle{License violation prevention}

The best way to solve a license violation is to have never have it happen in the first place!

There are two components:

\begin{itemize}
\item technical process improvements
\item non-technical process improvements
\end{itemize}
}

\frame{
\frametitle{Technical process improvements}

\begin{itemize}
\item check releases (source code and binary) from upstream. Trust, but verify!
\item provide compliant releases downstream
\item integrate compliance into your development process
\item actively teach your engineers about licensing and when it is OK to combine/reuse software
\item push your changes to upstream projects, so you cannot accidentily forget to include them the next time (plus all the other advantages!)
\end{itemize}
}

\frame{
\frametitle{Non-technical process improvements}

\begin{itemize}
\item push compliance upstream (if you source from companies) through contracts
\item demand a ``software bill of materials'' in SPDX (Software Package Data Exchange) or another format and verify it with a checker
\item check and record every release that is distributed, including beta/test versions
\item use SPDX (Software Package Data Exchange) or another format to clearly indicate what you are shipping
\end{itemize}
}

\frame{
\frametitle{Q\&A}
Remember: no questions about software patents
}

\frame{
\frametitle{Contact}

\begin{itemize}
\item \url{armijn@gpl-violations.org} for \url{gpl-violations.org} related questions
\item \url{info@tjaldur.nl} for consultancy questions
\end{itemize}

or just talk to me in the hallway.
}

\end{document}
