\documentclass[11pt]{beamer}

\usepackage{url}
\usepackage{tikz}
\author{Till Jaeger - JBB Rechtsanw\"alte\\Armijn Hemel - Tjaldur Software Governance Solutions}
\title{}
\title{OSADL License Compliance Audit}
\date{April 10, 2014}

\begin{document}

\setlength{\parskip}{4pt}

\frame{\titlepage}

\frame{
\frametitle{About OSADL}
The Open Source Automation Development Lab (OSADL) is a registered cooperative under German law
aimed at reducing cost for its members (43 regular members, 19 academic members, 5 sponsoring members)
while solving shared problems:

\begin{itemize}
\item (real time) kernel development
\item testing infrastructure
\item research into safety critical systems certification
\item legal issues
\end{itemize}

If you are interested in knowing more/joining we have brochures.
}

\frame{
\frametitle{Product fields}
OSADL members are in fields that are different from for example consumer
electronics (or making ``Flappy Bird'' clones):

\begin{itemize}
\item factory/building automation
\item lasers
\item robots
\item washing machines (normal ones, but also for DNA)
\item woodwork machines
\item agriculture
\item etc.
\end{itemize}

Many products are not flashy, but perform key roles in infrastructure.
}

\frame{
\frametitle{Considerations OSADL member companies have}
Products have a long life. Some have a ``return on investment'' of 20 years.
Support contracts often are (at least) as long. Facebook will be long gone by
then!!

Software upgrades are frequently very hard, or almost impossible to do, or
could be very risky, costly or even dangerous.

Many of these companies are therefore, by nature, risk averse.

Linux and Open Source helps them reduce this risk.
}

\frame{
\frametitle{OSADL license compliance audit}
The license compliance audit (LCA) was developed because OSADL member companies requested it.

The LCA is limited to a single device/firmware revision combination.

Two components on the system are audited:

\begin{enumerate}
\item Linux kernel including external kernel modules
\item C library (glibc/uClibc), including how other components are linked to it
\end{enumerate}

The audit is restricted to license compliance: detecting code theft/plagiarism,
patent issues and license compatibility of user space programs are not part of
the audit.

After successful completion of the audit the company gets a certificate.
}

\frame{
\frametitle{}
  \begin{tikzpicture}[remember picture,overlay]
    %\node[yshift=0.3cm] at (current page.center)
    \node[yshift=-2.5cm] at (current page.center)
    {
      \pgfimage[width=\linewidth]{OSADL-LCA-Certificate-by3.png}
    };
  \end{tikzpicture}
}

\frame{
\frametitle{Why did OSADL start this audit?}
A few practical reasons:

\begin{itemize}
\item many OSADL companies are (originally) not software companies, a lot of
knowledge is missing, yet open source is their (software) future
\item wrong choices made now could have effects lasting decades
\end{itemize}

The actual goal for companies is not the certificate, but getting a ``sanity
check'' and a better insight how well they are doing.

Also: no one had dared doing this before, so there was a challenge :-)
}

\frame{
\frametitle{Compliance audit setup}
The audit is on site and lasts a full day. The company has one (or more)
engineers involved in making the product available to answer any questions.

Furthermore we ask for:

\begin{itemize}
\item source code
\item binary image
\item documentation
\end{itemize}

The audit is done in a continuous dialogue and discussion between lawyer, audit
engineer and company engineers to ensure knowledge transfer from the auditors
to the company.
}

\frame{
\frametitle{Report}
For each audit an extensive report with the conclusions of the audit is made.
In the report problems found are described, plus steps that should be taken to
remedy these problems.
}

\frame{
\frametitle{Questionnaire}
Before we do an audit we ask companies to fill in a questionnaire:

\begin{itemize}
\item gives us some idea of what to expect
\item ensures that we have all necessary information to actually do an audit
\item forces companies to think about issues in advance
\end{itemize}
}

\frame{
\frametitle{Legal work done during the audit}

\begin{enumerate}
\item license text review (compatibility, completeness)
\item contract \& EULA review
\item written offer review
\item discussion about alternate ways to carry out license obligations
\end{enumerate}

Many licenses require that license texts and author information are present.
The review is to catch if licenses are missing or incomplete.

Contracts and EULA are checked for compatibility to see if they are compatible
with GPL/LGPL.

The written offer (if any) is reviewed to see if it is correct.
}

\frame{
\frametitle{Technical work done during the audit}

\begin{enumerate}
\item license scan
\item linking analysis
\item rebuild of kernel + C library/toolchain
\end{enumerate}
}

\frame{
\frametitle{First: reducing the search space}
Even though scope is limited it is still of work: recent Linux kernels have
around 40,000 files.

Looking at each file would be very costly (at 1 second per file it would be 11+
hours), so we take a few shortcuts.

\begin{enumerate}
\item keep a database with all files in upstream \url{kernel.org} kernels
\item determine a checksum for each file in the kernel we audit
\item if the file can be found in the database we trust it and ignore it
\item look in more detail at files that can't be found in the database
\end{enumerate}

For the Linux kernel this typically eliminates between 95\% (but usually closer
to 98\%) of the files and allows us to focus on the \textit{real} problems.

We think this is justified as the Linux kernel has already done due dilligence
on code it publishes. Similar: FSF (glibc) and SFC (uClibc).
}

\frame{
\frametitle{License scan}
Each file that could not be found in the database (a few hundred files) are
scanned with Nomos (part of FOSSology) and Ninka. This is to easily spot
problematic files and prioritize work.

The license and copyright notices of each of these files is also verified manually.
}

\frame{
\frametitle{Linking analysis and kernel/C library rebuild}
Using the Binary Analysis Tool (BAT) linking relationships between components
are researched and visualised.

We also ask the company's engineers to rebuild the kernel and C library (or
entire toolchain) on the spot and we compare the results with binaries on the
device.
}

\frame{
\frametitle{Common issues found: technical}
\begin{itemize}
\item missing source code
\item incorrectly licensed source code
\end{itemize}
}

\frame{
\frametitle{Common issues found: legal}
\begin{itemize}
\item contradicting statements in licenses and contracts
\item LGPLv2.1 section 6 (requiring to allow reverse engineering and relinking)
\end{itemize}
}

\frame{
\frametitle{Common issues found: organisational}
\begin{itemize}
\item tasks assigned to the wrong staff (who is responsible,
who handles source code requests, etc.)
\item purchase department does not take OSS licensing into account
\item legal documents are not made or reviewed by the legal department, but
by the development department (sometimes more than one)
\end{itemize}
}

\frame{
\frametitle{In conclusion: our experience}
We have found this type of audit to be the most effective ``tool'' available
to help achieve license compliance.

Although it is in essence a \textit{product} audit we uncover quite a few
\textit{process} defects: many problems tend to be shared between multiple
products.

In the future we also want to provide pure process audits as well as supplier audits.

If you want an audit, come talk to us! :-)
}

\frame{
\frametitle{Q\&A}
}

\end{document}
