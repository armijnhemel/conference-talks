\documentclass[11pt]{beamer}

\usepackage{url}
\usepackage{tikz}
\author{Armijn Hemel\\Tjaldur Software Governance Solutions}
\title{Simplifying compliance by trusting upstream}
\date{May 30, 2013}

\begin{document}

\setlength{\parskip}{4pt}

\frame{\titlepage}

\frame{
\frametitle{About Armijn}

\begin{itemize}
\item using Open Source software since 1994
\item MSc Computer Science from Utrecht University (The Netherlands)
\item core team \url{gpl-violations.org} from 2005 - May 2012
\item volunteer at \url{gpl-violations.org}
\item Tjaldur Software Governance Solutions since May 2011
\item Binary Analysis Tool
\end{itemize}
}

\frame{
\frametitle{Current day issues in consumer electronics}

There are plenty of GPL license issues in current day consumer electronics:

\begin{itemize}
\item no source code
\item partial source code
\item wrong source code
\item no license text/written offers
\end{itemize}

This has lead to license enforcement in various countries (Germany, France, US).
}

\frame{
\frametitle{Origin of license issues: supply chain}
Supply chains can be long: multiple companies are involved in creating a single product.

ODMs and (some) chipset vendors are notoriously sloppy: code (source \textit{and} binary) is pulled from many places (upstream projects, other suppliers, homebrew) and distributed without verifying license compliance.

Various reasons:

\begin{itemize}
\item market pressure
\item time to market
\item ignorance
\item ``competitive advantage''
\item low chance of getting caught
\end{itemize}
}

\frame{
\frametitle{Trust, but verify}
While business relationships should be built on trust it never hurts to \textit{verify}.

The large amount of files in source code releases makes it hard, even impractical, to do a full source code audit in little time.
}

\frame{
\frametitle{Some numbers}
\begin{itemize}
\item BusyBox 1.21.0: 1700+ files (771 C files, 97 header files)
\item Linux kernel 2.6.32.9 (used in Android Froyo): 30,000+ files
\item Linux kernel 3.0.31 (used in Android Jelly Bean): 36,000+ files
\item Linux kernel 3.8.4: 41,000+ files, (17997 C files, 15041 header files, 1306 assembler files)
\item Qt 5.0.1: 62,000+ files, (674 C files, 12013 C++ files, 12930 header files)
\end{itemize}

For an Android phone you are easily looking at millions of files in total.
}

\frame{
\frametitle{More numbers}
Lines of code (including comments and whitespace) in \textit{just} the C files:

\begin{itemize}
\item BusyBox 1.21.0: 255,687 lines of code
\item Linux kernel 3.8.4: 12,165,051 lines of code
\end{itemize}

%If automated scanning would cost USD 0.001 per line: over USD 12,000 for the Linux kernel (excluding header files)
}

\frame{
\frametitle{Learning from the LTSI project}
At LinuxCon Europe 2011 the LTSI project was launched. It introduced the``Yaminabe project'' to detect duplicate work. Yaminabe used a supersimple algorithm:

\begin{itemize}
\item treat all known files from upstream kernel as uninteresting
\item focus on differences
\end{itemize}

Using this method it was clear that around 95\% of sources in the kernel that vendors shipped was identical to upstream and could be ignored.

This same method can also be used for compliance scanning!
}

\frame{
\frametitle{My compliance method}
One very simple assumption:

Upstream projects (Linux kernel, BusyBox, Samba, \dots) are ``trusted'': whatever they publish is safe. Instead, focus on the files that are \textit{not} in upstream projects!

This is not about finding copied code in files: use a clone detector for that. However, my approach can be used as a very effective pre-filter.
}

\frame{
\frametitle{Why trust upstream projects?}

\begin{itemize}
\item they are the copyright holders
\item they chose the license
\item they can sue for copyright infringement
\item code from upstream software projects is often used (largely) unmodified
\end{itemize}

Using releases from upstream projects as the ``reference standard'' and only looking at differences vastly simplifies compliance.

Not every upstream source is trustworthy: you have to decide for yourself.
}

\frame{
\frametitle{High trust example: Linux kernel}
\begin{itemize}
\item every commit is ``signed off'' by multiple developers: high levels of scrutiny
\item for (almost) every line of code there is a very clear history
\item code that is not clear license wise has been in the kernel for 20 years and often copyrighted by very visible developers (Linus Torvalds, Ted Ts'o, etcetera)
\item high turnover of code: old code is gradually being replaced by code that is clearer license wise
\end{itemize}
}

\frame{
\frametitle{High trust example: BusyBox}
\begin{itemize}
\item there have been very visible legal cases about BusyBox
\item extremely high levels of scrutiny: they cannot afford a licensing mistake themselves
\end{itemize}

Of course you still need to distribute everything license compliant.
}

\frame{
\frametitle{High trust example: GNU project}
\begin{itemize}
\item there have been very visible legal cases about GNU project software
\item extremely high levels of scrutiny: they cannot afford a licensing mistake themselves
\item they are the ``moral standard''
\end{itemize}
}

\frame{
\frametitle{Low/no trust example: Maven 2 Central Repository}
Maven is a build tool for Java, that can use repositories for storing/getting binary artefacts. The Maven 2 Central Repository is awful:

\begin{itemize}
\item used as a dumping ground for Java code
\item tons of copies of the same code, with only slight modifications, unclear copyrights, unclear origins
\item thousands of companies trust it blindly!
\end{itemize}

Copyright time bomb waiting to explode!
}

\frame{
\frametitle{Others}

\begin{itemize}
\item Samba: high trust
\item GNOME: high trust (on their own code)
\item KDE: high trust (on their own code)
\item OpenWrt: low trust
\item Debian: low trust
\item Fedora: low trust
\end{itemize}

Rule of thumb: upstream sources that repackage source code from others (like Linux distributions) should have ``low trust''. Original authors should have ``high trust''. Roles are not always clear and different situations might require different levels of trust.
}

\frame{
\frametitle{Approach}

\begin{enumerate}
\item create a database from source code from trusted upstream. Record SHA256 checksum, plus possibly strings, function names, etcetera
\item unpack source code archive from untrusted supplier
\item compute SHA256 checksums for each source code file from the untrusted supplier
\item compare to checksums in database
\item ignore files that can be found in the database
\item further analyse/scan the files that are different
\end{enumerate}

That's it!
}

\frame{
\frametitle{Effectiveness}
This turns out to be \textit{extremely} effective:

\begin{itemize}
\item scans are fast: around 10 minutes on my test machine (8 threads, SSD) for a recent kernel. This includes a license scan with Ninka and FOSSology for files that could not be found in any upstream Linux kernel version.
\item Yaminabe project: scanning kernel releases of various Android handsets showed that between 5\% and 10\% of files of the Linux kernel differ. That is a reduction of the problem space with 90\% or more!
\end{itemize}

Audits are no longer impossible, but become very doable. I have used this successfully for multiple customers.
}

\frame{
\frametitle{Caveats}

\begin{itemize}
\item it requires a ``leap of faith''
\item not every upstream source is worth trusting in every circumstance
\item creating a complete enough database of good quality is \textit{very} very time consuming
\item database has to be kept up to date
\item does not check for \textit{all} compliance requirements (configuration files, etcetera)
\end{itemize}

Even if you would not trust upstream code, it is a great approach to \textit{prioritize} potential issues and risks.
}

\frame{
\frametitle{Method in practice}
I have used this method in practice:

\begin{itemize}
\item audits for Open Source Automation Development Lab (OSADL)
\item other customers
\end{itemize}
}

\frame{
\frametitle{Example case study: Linux based router}
%Source code archive from a router (Atheros chipset) made by TP-Link, a well known Chinese router manufacturer.
Source code archive from a router made by a well known Chinese router manufacturer.

Device uses many standard components, like:

\begin{itemize}
\item Linux kernel
\item BusyBox
\item U-Boot
\end{itemize}
}

\frame{
\frametitle{Linux kernel}
24987 files in the Linux kernel were scanned.

Of those slightly over 900 (about 4\%) could not be found in any kernel downloaded from \url{kernel.org}. 7 files could be found in kernels from other sources.

\begin{itemize}
\item version numbers, id, author fields commonly from CVS/RCS metadata by Perforce (likely)
%\item proprietary files (mostly Atheros, some TP-Link), likely problematic licensewise.
\item proprietary files (mostly chipset manufacturer, some device manufacturer), likely problematic licensewise.
\item actual changed source code for (perceived) bugs
\end{itemize}

Over 600 files were just changes made by version control system, meaning only a handful of files were \textit{actually} changed or new!
}

\frame{
\frametitle{BusyBox}
442 files were scanned, 62 (14\%) not in database:

\begin{itemize}
\item 17 files with Perforce changes
\item (modified) copy of bridge-utils
\item 1 proprietary chipset vendor file
%\item 1 proprietary Atheros file
\item 1 proprietary device manufacturer file
%\item 1 proprietary TP-Link file
\item changed files with patches for perceived bugs
\end{itemize}
}

\frame{
\frametitle{U-Boot}
2989 files were scanned, 395 (13\%) not in database, 10 not found in official U-Boot upstream:

\begin{itemize}
\item many chipset vendor proprietary files
%\item many Atheros proprietary files
\item many files with Perforce changes
\item several non-upstream files with different licenses and copyrights
\item 1 file undistributable for commercial use
\end{itemize}
}

\begin{frame}[fragile]
\frametitle{Undistributable file in U-Boot in router sources}
\texttt{u-boot/fs/jffs2/compr\_lzari.c} contains the following license:
\begin{verbatim}
LZARI.C (c)1989 by Haruyasu Yoshizaki, Haruhiko Okumura,
and Kenji Rikitake.

All rights reserved. Permission granted for
non-commercial use.
\end{verbatim}

Luckily this file is not used in the binary!
\end{frame}

\frame{
\frametitle{Tooling}
Scripts to create database are already present in Binary Analysis Tool (BAT).

Basic script to scan will be released as part of BAT 15 (already present in version control).

Fancier programs with more functionality will be offered as a service.
}

\frame{
\frametitle{Questions?}
}

\frame{
\frametitle{Contact}
Any more questions? Feel free to contact me!

\begin{itemize}
\item \url{armijn@tjaldur.nl}
\item \url{http://www.tjaldur.nl/}
\item Binary Analysis Tool: \url{http://www.binaryanalysis.org/}
\end{itemize}
}

\end{document}
