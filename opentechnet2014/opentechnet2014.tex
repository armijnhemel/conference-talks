\documentclass[11pt]{beamer}

\usepackage{url}
\usepackage{tikz}
\usepackage{kotex}
\author{헤멜 아마인\\Armijn Hemel, MSc\\Tjaldur Software Governance Solutions}
\title{}
\date{November 5, 2014}

\begin{document}

\setlength{\parskip}{4pt}

\frame{\titlepage}

\frame{
\frametitle{About Armijn}

\begin{itemize}
\item using Open Source software since 1994
\item MSc Computer Science from Utrecht University (The Netherlands)
\item core team \url{gpl-violations.org} from 2005 - May 2012, helping solve hundreds of incompliance cases
\item owner Tjaldur Software Governance Solutions, helping companies with compliance issues
\item creator of the Binary Analysis Tool for analysing binary artefacts for license and security issues
\item European coordinator Linux Defenders at Open Invention Network
\end{itemize}
}

\frame{
\frametitle{Today's topic}
Today I will talk about:

\begin{itemize}
\item why using open source makes business sense
\item enforcement of one open source license, namely GNU General Public License (GPL)
\item how to avoid getting in trouble with GPL enforcers
\end{itemize}

I am not a lawyer and nothing I say in this talk is legal advise. When in doubt, ask a competent lawyer.

Nothing I am saying today is rocket science, and everything is ``common sense''. This is because open source license trouble usually is not a technical problem but a \textit{process} problem.
}

\frame{
\frametitle{Software reuse is smart}
Software is increasingly commoditized as open source software. In the late 1980s it started with single software components, now entire software stacks are a commodity, like:

\begin{itemize}
\item OpenStack and Apache Cloudstack (server)
\item Android (mobile)
\item OpenWrt and Yocto (networking)
\end{itemize}

Not reusing software, but developing everything from scratch is becoming a \textit{stupid business decision}.
}

\frame{
\frametitle{Open Source}
Many components are available as ``Open Source''.

The source code for these components is available and the licenses allow you to reuse and redistribute, at (in practice) no cost. Many of these components are high quality (note: there are also many bad quality open source components!).

Needless to say that a lot of open source is used by many vendors: a few years ago Gartner predicted (some form of) open source would used in \textit{every} device or program in 2012.
}

\frame{
\frametitle{The various open source licenses}
There is no single ``Open Source''. There are about 60 open source licenses that are frequently used. They differ in various ways:

\begin{itemize}
\item distribution under the same (or stricter) license of ``derived software'', or offer option to make code ``closed'' again
\item (no) patent language
\item (no) attribution needed
\item applies for network distribution (or not)
\end{itemize}

Not all the licenses are compatible and you need to take care of following the license terms correctly. This is not always trivial and you might not be able to combine software packages \textit{legally}. In practice it is straightforward for most software packages.

Today I will (mostly) focus on the GNU licenses.
}

\frame{
\frametitle{GNU license family}
One of the most popular open source license families is the GNU license family. It contains:

\begin{itemize}
\item General Public License (GPL)
\item Lesser General Public License (LGPL) -- originally intended for shared libraries
\item Affero General Public License (AGPL) -- networked services
\end{itemize}

Not every version of every GNU license is compatible with other licenses. Example: GPL 2 and GPL 3 are not compatible.
}

\frame{
\frametitle{GNU General Public License}
There are various versions of the GNU General Public License, with GPL 3 being the latest. Some widely used components licensed under GPL are:

\begin{itemize}
\item Linux kernel: GPL 2
\item BusyBox: GPL 2
\item Samba (before 3.2): GPL 2 or later
\item Samba (3.2 and later): GPL 3 or later
\end{itemize}

GPL grants you a lot of freedom, in exchange for keeping the code and modifications open under the same license.

Failure to comply with the license conditions of the GPL license renders the license void. You could then be sued by copyright holders for a \textit{copyright violation}.
}

\frame{
\frametitle{GPL enforcement around the world}
The GPL license has been enforced in several countries:

\begin{itemize}
\item Germany (several law suits, hundreds of cases settled before going to court)
\item USA (settled before going to court)
\item France (settled before going to court)
\end{itemize}

During a case you might not be able to sell your product in a particular country for a significant period of time.

I have been involved in cases in Germany and the US, on both sides.
}

\frame{
\frametitle{Known license enforcement}
Some of the packages I know the license was enforced for:

\begin{itemize}
\item Linux kernel
\item BusyBox
\item XviD
\item U-Boot
\item GNU utilities
\item Samba
\item several more programs
\end{itemize}

None of the cases so far were primarily about \textit{money} but about \textit{license compliance}. Not solving issues can be costly though.
}

\frame{
\frametitle{GPL license violations are a business risk}
The risk of getting sued by copyright holders exists, but might not be very high. You might get away for quite a long time. But GPL license violations are still a business risk!

If your exit strategy is to be bought by a big company remember they \textit{will} (or \textit{should}) scan everything. If you have license problems you are a lot less attractive for them: no one wants to buy a liability.
}

\frame{
\frametitle{Preventing (GPL) license violations: 어떻게?}

\begin{itemize}
\item know your limitations
\item know what the GPL license demands
\item know what you use
\item document what you use
\item document your changes
\item rebuild your software in a clean environment
\item upstream changes as much as possible
\end{itemize}
}

\frame{
\frametitle{Know your limitations}
Many engineers approach licenses as if they were mathematics. But licenses are legal documents and law is not math!

Legal documents describe interactions between humans and codify \textit{expectations}. Software licenses are therefore inherently human. Humans are not math! There is always a gray area, with uncertainty.
}

\frame{
\frametitle{Know what the GPL license demands}
You should read the actual license text before drawing any conclusions. When in doubt talk to a knowledgeable lawyer.

\begin{itemize}
\item complete \& corresponding source code with the product, or
\item written offer for the complete \& corresponding source code, valid for three years
\end{itemize}

Complete \& corresponding source code: everything needed to recreate the binary file, and licensed under GPL or a GPL compatible license.
}

\frame{
\frametitle{Know what you use}
We are living in the ``golden age of plagiarism''!

You need to know what you use and how you integrate it. Scan incoming components for possible license issues:

\begin{itemize}
\item closed source tooling (licenses/copyright/code copying)
\item FOSSology (licenses/copyright)
\item Ninka (licenses)
\item Binary Analysis Tool (for binary only components from suppliers)
\item or a combination of these tools
\end{itemize}

Be careful with copy/paste from unknown sources (webpage, discussion forum, etc.)

Also discuss with your colleagues what you import and why.
}

\frame{
\frametitle{Document what you use}
The GPL license requires ``an appropriate copyright notice''. It is not exactly clear what this means.

At least one copyright holder doing enforcement actively checks if there is a list of components and if it is complete as part of his compliance checks.

Document clearly what you use. Adapt your build system so it generates SPDX documents.

SPDX: Software Package Data eXchange, a format to specify a ``software bill of materials''
}

\frame{
\frametitle{Document your changes}
The GPL license says that ``modified files must carry prominent notices
stating that you changed the files and the date of any change''. This is often neglected.

Keep track what you change. If not in the file, then at least use meaningful commit messages for version control!
}

\frame{
\frametitle{Rebuild your software in a clean environment}
Many companies do not rebuild their software in a clean environment. People who enforce the license typically \textit{do}.

Often there are undocumented dependencies or configuration options. A rebuild in a different environment often fails.

By testing if your software builds in a \textit{clean} environment according to the instructions you provide you can catch many problems before others do.

If rebuilding requires information from the build system/scripts, then you should add those too as part of ``complete \& corresponding source code''.

You really should consider using standardized open source build systems (OpenWrt, Yocto, Android, etcetera)
}

\frame{
\frametitle{Upstream your changes}
One reason that many GPL violations get noticed is because products have additional functionality (like hardware support) that is not in the ``upstream'' version.

People want to study, share and improve the \textit{new} functionality, but if there is no source code they get mad. Upstreaming makes you a better ``open source citizen''. It also makes it easier to contact you in case there are issues.
}

\frame{
\frametitle{Dealing with (GPL) license violations: 어떻게?}

\begin{itemize}
\item communicate
\item participate in the global discussion about compliance
\item learn \& teach
\end{itemize}
}

\frame{
\frametitle{Communicate}
Some people doing enforcement get very irritated if companies do not respond or do not communicate well. They get emotional and irrational and assume you are violating the license on purpose and will try to catch you for even the smallest things because you are ``bad''.

Communication (coordinate with your legal team!) is key.
}

\frame{
\frametitle{Participate in the global discussion about compliance}
When dealing with compliance you are not alone. There are several groups discussing licensing issues and you are welcome to participate:

\begin{itemize}
\item Free Software Foundation Europe runs the ``legal network'' with hundreds of experts worldwide and organises an (invite-only) legal conference each April (mid-April 2015, Barcelona)
\item Linux Foundation frequently has legal talks at its conferences
\item Linux Foundation Japan organises an (invite-only) legal conference each October or November (13/14 November 2014, Yokohama)
\item KOSS Law Center organises FOSSCon Korea each year (4/5 December 2014, Seoul)
\end{itemize}

but also conferences like today.
}

\frame{
\frametitle{Learn \& teach}
Spreading knowledge inside your company is very important: you don't want a ``single point of failure'' for compliance issues.

Talk to people, teach (new) colleagues, build up local expertise.

Ideally this should actually start before people start to work at your company.
}

\frame{
\frametitle{Push for basic copyright classes at college/university}
Open source ideas are moving into other domains, such as data, design, art, biochemistry and medicine. Copyright licensing will become much more important.

My request to NIPA: please push for having a mandatory basic copyright course at college/university. This would get Korea ahead of other countries. In Europe these are often not given, or optional as part of a master's degree.
}

\frame{
\frametitle{Questions?}
}

\frame{
\frametitle{Contact}

\begin{itemize}
\item \url{armijn@tjaldur.nl}
\item \url{http://www.tjaldur.nl/}
\item Binary Analysis Tool: \url{http://www.binaryanalysis.org/}
\end{itemize}
}

\end{document}
